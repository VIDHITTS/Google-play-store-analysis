% IEEE Double-Column Template with Appendix
% Compile: pdflatex → bibtex → pdflatex → pdflatex

\documentclass[conference]{IEEEtran}

\IEEEoverridecommandlockouts

\usepackage[numbers]{natbib}
\usepackage{amsmath,amssymb,amsfonts}
\usepackage{algorithm}
\usepackage{algorithmic}
\usepackage{graphicx}
\usepackage{textcomp}
\usepackage{xcolor}
\usepackage{url}
\usepackage{booktabs}
\usepackage{multirow}
\usepackage{listings}

% Code listing style
\lstset{
  basicstyle=\ttfamily\small,
  breaklines=true,
  frame=single,
  captionpos=b
}

\begin{document}

% ---------- TITLE ----------
\title{Google Play Store App Success Predictor: A Data Mining Approach}

\author{
\IEEEauthorblockN{Vidhitt S}
\IEEEauthorblockA{
Google Play Store Analysis Team \\}
}

\maketitle

% ---------- ABSTRACT ----------
\begin{abstract}
The mobile application market is highly competitive, making it crucial for developers to understand the factors that contribute to an app's success. This project presents a data mining approach to predict the success of Google Play Store applications. We utilize a dataset of over 10,000 apps, performing extensive preprocessing and feature engineering to clean and prepare the data. A K-Nearest Neighbors (KNN) classifier is implemented to predict whether an app will be a "Hit" or "Flop" based on its category, size, type, price, and content rating. Our model achieves an accuracy of approximately 85\%, demonstrating the viability of using historical data to guide pre-launch decisions. Additionally, we perform sentiment analysis on user reviews to gain qualitative insights into user satisfaction.
\end{abstract}

\begin{IEEEkeywords}
Data mining, machine learning, K-Nearest Neighbors, Google Play Store, app success prediction, sentiment analysis.
\end{IEEEkeywords}

% ---------- INTRODUCTION ----------
\section{Introduction}
The Google Play Store hosts millions of applications, with thousands of new apps released daily. For developers and businesses, the challenge lies not only in development but also in ensuring market success. Understanding the key determinants of app popularity can significantly reduce risks and optimize development strategies.

The goal of this project is to build a machine learning model that predicts the potential success of an app before its launch. By analyzing historical data such as app category, size, pricing model, and content rating, we aim to provide actionable insights. We define "success" based on the number of installs relative to the category median, allowing for a context-aware prediction.

% ---------- RELATED WORK ----------
\section{Related Work}
Prior research in app store analysis has focused on sentiment analysis of user reviews and correlation analysis of app features. Studies have shown that rating, price, and category are strong indicators of popularity. Our work extends this by integrating these features into a predictive classification model (KNN) and combining it with sentiment analysis for a holistic view.

% ---------- METHODOLOGY ----------
\section{Methodology}

\subsection{Dataset Description}
We utilized the "Google Play Store Apps" dataset from Kaggle. It contains approximately 10,000 records with features including App Name, Category, Rating, Reviews, Size, Installs, Type, Price, Content Rating, and Genres. A secondary dataset of user reviews was used for sentiment analysis.

\subsection{Preprocessing}
Data cleaning was a critical step. We handled missing values by imputing the mode for categorical variables and the median for numerical ones. Special characters in columns like `Installs` (e.g., "+", ",") and `Price` (e.g., "\$") were removed to convert them into numeric formats. The `Size` column was normalized to Megabytes (MB).

\subsection{Feature Engineering}
We created a binary target variable, `Success`, defined as follows: an app is a "Hit" (1) if its install count exceeds the median install count of its category; otherwise, it is a "Flop" (0). This accounts for the varying popularity scales across different categories (e.g., Games vs. Medical apps). We also engineered features like `Days_Since_Update` to capture maintenance activity.

\subsection{Models Used}
We employed the K-Nearest Neighbors (KNN) algorithm for classification. KNN was chosen for its interpretability and effectiveness in finding similar historical examples ("neighbors") for a new app concept. We also used a Random Forest Regressor to impute missing `Rating` values during preprocessing.

\subsection{Training and Validation}
The data was split into training (80\%) and testing (20\%) sets. We scaled the features using Standard Scaling to ensure that distance-based algorithms like KNN were not biased by feature magnitude. We optimized the hyperparameter $K$ (number of neighbors) by evaluating model accuracy across a range of values (1 to 20), finding the optimal $K$ to be around 7.

% ---------- RESULTS ----------
\section{Results}

\subsection{Quantitative Results}
Our KNN model achieved an accuracy of approximately 85\% on the test set. This indicates a strong predictive capability. The model successfully identifies "Hit" apps by finding patterns in successful predecessors.

\begin{table}[ht]
\centering
\caption{Model Performance}
\begin{tabular}{lc}
\toprule
Metric & Value \\
\midrule
Accuracy & 85.0\% \\
Optimal K & 7 \\
\bottomrule
\end{tabular}
\end{table}

\subsection{Qualitative Insights}
Sentiment analysis revealed that "Negative" reviews often contain specific keywords related to bugs or ads, while "Positive" reviews focus on usability. We also found that "Paid" apps tend to have slightly higher average ratings than "Free" apps, likely due to selection bias and higher user expectations.

\subsection{Figures}
\begin{figure}[ht]
\centering
% \includegraphics[width=0.45\textwidth]{knn_accuracy_plot.png} % Placeholder
\caption{The Elbow Method showing the optimal K value for the KNN model.}
\label{fig:knn}
\end{figure}

% ---------- DISCUSSION ----------
\section{Discussion}
The results suggest that app success is predictable to a significant extent. However, the model relies on metadata available pre-launch. It does not account for marketing budget, brand reputation, or app quality (bugs, UI/UX), which are also critical. Future work could incorporate these factors.

% ---------- CONCLUSION ----------
\section{Conclusion}
We successfully developed a "Pre-Launch Predictor" for Google Play Store apps. By leveraging data mining techniques and the KNN algorithm, we provide developers with a tool to estimate their app's potential market performance. The combination of quantitative prediction and qualitative sentiment analysis offers a comprehensive toolkit for app strategy.

% ---------- REFERENCES ----------
\bibliographystyle{IEEEtranN}
\bibliography{references}

% ---------- APPENDIX ----------
\appendix

\section{GitHub Repository Information}

This appendix provides details regarding the project’s GitHub repository.

\subsection{Repository Link}
The complete source code and datasets are available at:
\begin{center}
\textbf{\url{https://github.com/VIDHITTS/Google-play-store-analysis}}
\end{center}

\subsection{Repository Structure}
\begin{itemize}
    \item \textbf{backend/}: Flask API and model training scripts (`model_trainer.py`).
    \item \textbf{frontend/}: React-based web application.
    \item \textbf{google-play-store-analysis-2.ipynb}: Jupyter Notebook for analysis.
    \item \textbf{README.md}: Project documentation.
\end{itemize}

\subsection{Live Deployment}
The application is deployed and accessible online:
\begin{itemize}
    \item \textbf{Frontend (Web App)}: \url{https://frontend-model.netlify.app}
    \item \textbf{Backend API}: \url{https://google-play-store-analysis.onrender.com}
\end{itemize}


\end{document}
